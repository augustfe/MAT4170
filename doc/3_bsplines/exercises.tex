\section{B-splines}

\begin{exercise}
    Suppose that $x_0, x_1, x_2$ are distinct, and let $f_i = f(x_i)$, $i = 0, 1, 2$, for some function $f$.
    Show by direct calculation that the recursive formula
    \begin{equation*}
        [x_0, x_1, x_2]f =
        \frac{
            \frac{f_2 - f_1}{x_2 - x_1}
            - \frac{f_1 - f_0}{x_1 - x_0}
        }{x_2 - x_0}
    \end{equation*}
    can be expressed as
    \begin{equation*}
        [x_0, x_1, x_2]f = \sum_{i=0}^{2} \frac{f_i}{\prod_{j \neq i} (x_i - x_j)}.
    \end{equation*}
\end{exercise}

\begin{solution}
    We have
    \begin{align*}
        [x_0, x_1, x_2]f
        &= \frac{
            \frac{f_2 - f_1}{x_2 - x_1}
            - \frac{f_1 - f_0}{x_1 - x_0}
        }{x_2 - x_0}
    \end{align*}
    where we begin by expanding the top fractions.
    \begin{align}
        \frac{f_2 - f_1}{x_2 - x_1} - \frac{f_1 - f_0}{x_1 - x_0}
        &= \frac{f_2}{x_2 - x_1} - f_1 \left( \frac{1}{x_2 - x_1} + \frac{1}{x_1 - x_0} \right) + \frac{f_0}{x_1 - x_0} \nonumber \\
        &= \frac{f_2}{x_2 - x_1} - f_1 \frac{x_1 - x_0 + x_2 - x_1}{(x_2 - x_1)(x_1 - x_0)} + \frac{f_0}{x_1 - x_0} \nonumber \\
        &= \frac{f_2}{x_2 - x_1} - f_1 \frac{x_2 - x_0}{(x_2 - x_1)(x_1 - x_0)} + \frac{f_0}{x_1 - x_0} \nonumber \\
        &= \frac{f_2}{x_2 - x_1} + \frac{f_1(x_2 - x_0)}{(x_1 - x_2)(x_1 - x_0)} + \frac{f_0}{x_1 - x_0} \label{eq:top_fraction}
    \end{align}
    Dividing~\eqref{eq:top_fraction} by $x_2 - x_0$ gives
    \begin{align*}
        [x_0, x_1, x_2]f
        &= \frac{
            \frac{f_2 - f_1}{x_2 - x_1}
            - \frac{f_1 - f_0}{x_1 - x_0}
        }{x_2 - x_0} \\
        &= \frac{f_2}{(x_2 - x_1)(x_2 - x_0)}
        + \frac{f_1}{(x_1 - x_2)(x_1 - x_0)}
        + \frac{f_0}{(x_1 - x_0)(x_2 - x_0)} \\
        &= \frac{f_2}{(x_2 - x_1)(x_2 - x_0)}
        + \frac{f_1}{(x_1 - x_2)(x_1 - x_0)}
        + \frac{f_0}{(x_0 - x_1)(x_0 - x_2)} \\
        &= \sum_{i=0}^{2} \frac{f_i}{\prod_{j \neq i} (x_i - x_j)},
    \end{align*}
    as desired.
\end{solution}

\begin{exercise}
    Show that if $f(x) = 1/x$ and that $x_0, x_1, \ldots, x_k \neq 0$ then
    \begin{equation*}
        [x_0, \ldots, x_k]f = (-1)^k \frac{1}{x_0 x_1 \cdots x_k}.
    \end{equation*}
\end{exercise}

\begin{solution}
    In the base case we have simply
    \begin{equation*}
        [x_0]f = \frac{1}{x_0} = (-1)^0 \frac{1}{x_0}.
    \end{equation*}
    In the case of $x_0 = x_1 = \cdots = x_k$ we have
    \begin{equation*}
        [\underbrace{x_0, x_0, \ldots, x_0}_{k+1}]f = \frac{f^{(k)}(x_0)}{k!} = (-1)^k \frac{1}{x_0^{k+1}} \frac{k!}{k!} = (-1)^k \frac{1}{x_0 x_1 \cdots x_k},
    \end{equation*}
    so mulitplicities are handled correctly.
    For two distinct points $x_0, x_1$ we have
    \begin{equation*}
        [x_0, x_1]f = \frac{f_1 - f_0}{x_1 - x_0} = \frac{1/x_1 - 1/x_0}{x_1 - x_0} = \frac{x_0 - x_1}{x_0 x_1 (x_1 - x_0)} = (-1)^1 \frac{1}{x_0 x_1},
    \end{equation*}
    so the formula holds for $k = 1$.
    Assume that the formula holds for $k = n$, and consider $k = n + 1$.
    We have
    \begin{align*}
        [x_0, \ldots, x_{n+1}]f
        &= \frac{[x_1, \ldots, x_{n+1}]f - [x_0, \ldots, x_{n}]f}{x_{n+1} - x_0} \\
        &= \frac{(-1)^n \frac{1}{x_1 \cdots x_{n+1}} - (-1)^n \frac{1}{x_0 \cdots x_n}}{x_{n+1} - x_0} \\
        &= \frac{
            (-1)^n \frac{1}{x_{n+1}} - (-1)^n \frac{1}{x_0}
        }{
            (x_{n+1} - x_0) x_1 \cdots x_n
        } \\
        &= \frac{
            (-1)^n \frac{x_0 - x_{n+1}}{x_0 x_{n+1}}
        }{
            (x_{n+1} - x_0) x_1 \cdots x_n
        } \\
        &= (-1)^{n+1}\frac{
            x_{n+1} - x_0
        }{
            (x_{n+1} - x_0) x_0 x_1 \cdots x_n x_{n+1}
        } \\
        &= (-1)^{n+1}\frac{1}{x_0 x_1 \cdots x_{n+1}},
    \end{align*}
    proving the formula by induction.
\end{solution}

\begin{exercise}
    Prove the Leibniz rule for divided differences:
    \begin{equation*}
        [x_0, x_1, \ldots, x_k](fg) = \sum_{i=0}^{k} [x_0, \ldots, x_i]f [x_i, \ldots, x_k]g.
    \end{equation*}
    Hint: let $p$ and $q$ be the polynomials of degree $\leq k$ that interpolate $f$ and $g$ respectively at $x_0, x_1, \ldots, x_k$, and express $p$ and $q$ as
    \begin{align*}
        p(x) &= \sum_{i=0}^{k} (x - x_0) \cdots (x - x_{i-1})[x_0, \ldots, x_i]f, \\
        q(x) &= \sum_{j=0}^{k} (x - x_{j+1}) \cdots (x - x_k)[x_j, \ldots, x_k]g.
    \end{align*}
    Now consider the polynomial $pq$.
\end{exercise}

\begin{solution}
    Considering the hint, it seems important to consider the definition of divided differences based on the leading coefficient of the interpolating polynomial.
    Let $p$ and $q$ be the polynomials of degree $\leq k$ that interpolate $f$ and $g$ respectively at $x_0, x_1, \ldots, x_k$, and lets assume for now that $x_i \neq x_j$ for $i \neq j$.
    We write $p$ and $q$ as
    \begin{align*}
        p(x) &= \sum_{i=0}^{k} (x - x_0) \cdots (x - x_{i-1})[x_0, \ldots, x_i]f, \\
        q(x) &= \sum_{j=0}^{k} (x - x_{j+1}) \cdots (x - x_k)[x_j, \ldots, x_k]g,
    \end{align*}
    that is, in Newton form, where $q$ is written in reverse order.

    We now consider the polynomial $pq$.
    For brevity, we write $p$ and $q$ as
    \begin{equation*}
        p(x) = \sum_{i = 0}^k \prod_{a = 0}^{i - 1} (x - x_a) [x_0, \ldots, x_i]f,
        \quad
        q(x) = \sum_{j = 0}^k \prod_{b = j + 1}^{k} (x - x_b) [x_j, \ldots, x_k]g.
    \end{equation*}
    Then,
    \begin{align*}
        (p \cdot q)(x)
        &= \left( \sum_{i = 0}^k \prod_{a = 0}^{i - 1} (x - x_a) [x_0, \ldots, x_i]f \right) \left( \sum_{j = 0}^k \prod_{b = j + 1}^{k} (x - x_b) [x_j, \ldots, x_k]g \right) \\
        &= \sum_{i = 0}^k \sum_{j = 0}^k \prod_{a = 0}^{i - 1} (x - x_a) [x_0, \ldots, x_i]f \prod_{b = j + 1}^{k} (x - x_b) [x_j, \ldots, x_k]g \\
        &= \sum_{i = 0}^k \sum_{j = 0}^k [x_0, \ldots, x_i]f [x_j, \ldots, x_k]g \prod_{a = 0}^{i - 1} (x - x_a) \prod_{b = j + 1}^{k} (x - x_b).
    \end{align*}

    What about the unique polynomial of degree $\leq k$ that interpolates $fg$ at $x_0, x_1, \ldots, x_k$?
    We know that the polynomial must be of the form
    \begin{equation*}
        r(x) = \sum_{m = 0}^k [x_0, \ldots, x_m](fg) \prod_{n = 0}^{m - 1} (x - x_n),
    \end{equation*}
    and we know that $r(x)$ must be equal to $(p \cdot q)(x)$ at $x_0, x_1, \ldots, x_k$, as
    \begin{equation*}
        r(x_i) = f_i \cdot g_i = p(x_i) \cdot q(x_i) = (p \cdot q)(x_i),
        \quad i = 0, 1, \ldots, k.
    \end{equation*}

    Consider now a specific $x_\ell \in \{x_0, \ldots, x_k\}$.
    We have
    \begin{align*}
        r(x_\ell)
        &= \sum_{m = 0}^k [x_0, \ldots, x_m](fg) \prod_{n = 0}^{m - 1} (x_\ell - x_n) \\
        &= \sum_{m = 0}^\ell [x_0, \ldots, x_m](fg) \prod_{n = 0}^{m - 1} (x_\ell - x_n) + \sum_{m = \ell + 1}^k [x_0, \ldots, x_m](fg) \prod_{n = 0}^{m - 1} (x_\ell - x_n) \\
        &= \sum_{m = 0}^\ell [x_0, \ldots, x_m](fg) \prod_{n = 0}^{m - 1} (x_\ell - x_n),
    \end{align*}
    where the last equality follows as $\prod_{n = 0}^{m - 1} (x_\ell - x_n) = 0$ for $m > \ell$.
    Considering $p$ and $q$ seperately, we similarly have
    \begin{align*}
        p(x_\ell)
        &= \sum_{i = 0}^k [x_0, \ldots, x_i]f \prod_{a = 0}^{i - 1} (x_\ell - x_a) , \\
        &= \sum_{i = 0}^\ell [x_0, \ldots, x_i]f \prod_{a = 0}^{i - 1} (x_\ell - x_a) ,
    \end{align*}
    and
    \begin{align*}
        q(x_\ell)
        &= \sum_{j = 0}^k [x_j, \ldots, x_k]g \prod_{b = j + 1}^{k} (x_\ell - x_b) , \\
        &= \sum_{j = \ell}^k [x_j, \ldots, x_k]g \prod_{b = j + 1}^{k} (x_\ell - x_b) .
    \end{align*}

    In order to get a better intuition as to how to proceed, let's consider the points $x_0$.
    We have
    \begin{equation*}
        r(x_0) = [x_0](fg) = f_0 g_0
        \quad \text{and} \quad
        p(x_0) = [x_0]f = f_0,
    \end{equation*}
    while $q$ gives us
    \begin{equation*}
        q(x_0) =
        \sum_{j = 0}^k [x_j, \ldots, x_k]g \prod_{b = j + 1}^{k} (x_0 - x_b).
    \end{equation*}
    We already know that $q(x_0) = g_0$, however actually calculating everything seems tricky, likely involving some telescoping sum from the recursive formula.

\end{solution}