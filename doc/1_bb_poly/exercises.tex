\begin{exercise}
    It is sometimes necessary to convert a polynomial in BB form to monomial form.
    Consider a quadratic BB polynomial,
    \begin{equation*}
        p(x) = c_0 (1 - x)^2 + 2c_1 x(1 - x) + c_2 x^2.
    \end{equation*}
    Express $p$ in the monomial form
    \begin{equation*}
        p(x) = a_0 + a_1 x + a_2 x^2.
    \end{equation*}
\end{exercise}

\begin{solution}
    Rather than using the explicit formula for conversion, we can just expand the coefficients and collect terms.
    \begin{align*}
        p(x) &= c_0 (1 - x)^2 + 2c_1 x(1 - x) + c_2 x^2 \\
        &= c_0 (1 - 2x + x^2) + 2c_1 (x - x^2) + c_2 x^2 \\
        &= c_0 - 2c_0 x + c_0 x^2 + 2c_1 x - 2c_1 x^2 + c_2 x^2 \\
        &= c_0 + (-2c_0 + 2c_1) x + (c_0 - 2c_1 + c_2) x^2.
    \end{align*}
\end{solution}