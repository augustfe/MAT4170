\documentclass[
    a4paper,
    12pt,
]{article}
\usepackage{amsmath, amsthm, amsfonts, amssymb}
\usepackage{microtype}
\usepackage{geometry}
\usepackage{booktabs}
\usepackage{graphicx}
\usepackage{caption}
\usepackage{subcaption}
% \geometry{margin=1in}

\usepackage{hyperref}
\usepackage{xcolor}
\hypersetup{ % this is just my personal choice, feel free to change things
    colorlinks,
    linkcolor={red!50!black},
    citecolor={blue!50!black},
    urlcolor={blue!80!black},
}

\newtheoremstyle{exerciseStyle}
{ } % Space above
{ } % Space below
{ \normalfont } % Body font % chktex 1
{ } % Indent amount
{ \bfseries } % Theorem head font
{ } % Punctuation after theorem head
{ } % Space after theorem head
{\thmname{#1}\thmnumber{ #2}} % Theorem head spec (can be left empty, meaning `normal`)
\theoremstyle{exerciseStyle}
\newtheorem{exercise}{Exercise}[section]

\newtheoremstyle{solutionStyle}
{ } % Space above
{ } % Space below
{ \normalfont } % Body font % chktex 1
{ } % Indent amount
{ \bfseries } % Theorem head font
{ } % Punctuation after theorem head
{ } % Space after theorem head
{\thmname{#1}\thmnumber{ #2}} % Theorem head spec (can be left empty, meaning `normal`)
\theoremstyle{solutionStyle}
\newtheorem{solution}{Solution}[section]

\title{
    MAT4170\\
    \small{Exercises for Spline Methods}
}
\author{August Femtehjell}
\date{Spring 2025}

\begin{document}

\maketitle

\tableofcontents

\section{Bernstein-Bézier polynomials}

\begin{exercise}
    It is sometimes necessary to convert a polynomial in BB form to monomial form.
    Consider a quadratic BB polynomial,
    \begin{equation*}
        p(x) = c_0 (1 - x)^2 + 2c_1 x(1 - x) + c_2 x^2.
    \end{equation*}
    Express $p$ in the monomial form
    \begin{equation*}
        p(x) = a_0 + a_1 x + a_2 x^2.
    \end{equation*}
\end{exercise}

\begin{solution}
    Rather than using the explicit formula for conversion, we can just expand the coefficients and collect terms.
    \begin{align*}
        p(x) &= c_0 (1 - x)^2 + 2c_1 x(1 - x) + c_2 x^2 \\
        &= c_0 (1 - 2x + x^2) + 2c_1 (x - x^2) + c_2 x^2 \\
        &= c_0 - 2c_0 x + c_0 x^2 + 2c_1 x - 2c_1 x^2 + c_2 x^2 \\
        &= c_0 + (-2c_0 + 2c_1) x + (c_0 - 2c_1 + c_2) x^2.
    \end{align*}
\end{solution}

\begin{exercise}
    Consider a polynomial $p(x)$ of degree $\leq d$, for arbitrary $d$.
    Show that if
    \begin{equation*}
        p(x) = \sum_{j=0}^d a_j x^j = \sum_{i=0}^d c_i B_i^d(x),
    \end{equation*}
    then
    \begin{equation*}
        a_j = \binom{d}{j} \Delta^j c_0.
    \end{equation*}
    \textit{Hint:} Use a Taylor approximation to $p$ to show that $a_j = p^{(j)}(0)/j!$. % chktex 40
\end{exercise}

\begin{solution}
    We have that
    \begin{equation*}
        p(x) = \sum_{j=0}^d a_j x^j = \sum_{i=0}^d c_i B_i^d(x).
    \end{equation*}
    By the Taylor approximation, we have that
    \begin{equation*}
        p(x) = p(x + 0) = \sum_{j=0}^d \frac{p^{(j)}(0)}{j!} x^j.
    \end{equation*}
    We thus have that
    \begin{equation*}
        a_j = \frac{p^{(j)}(0)}{j!}.
    \end{equation*}
    By properties of the Bézier curves, we have that
    \begin{equation*}
        p^{(j)}(x) = \frac{d!}{(d-j)!} \sum_{i = 0}^{d - j} \Delta^j c_i B_i^{d-j}(x),
    \end{equation*}
    and specifically for $x = 0$,
    \begin{equation*}
        p^{(j)}(0) = \frac{d!}{(d-j)!} \Delta^j c_0.
    \end{equation*}
    Combining these results, we have that
    \begin{equation*}
        a_j = \frac{p^{(j)}(0)}{j!} = \frac{d!}{(d-j)! j!} \Delta^j c_0 = \binom{d}{j} \Delta^j c_0,
    \end{equation*}
    as we wanted to show.
\end{solution}

\begin{exercise}
    We might also want to convert a polynomial from monomial form to BB form.
    Using Lemma 1.2, show that in the notation of the previous question,
    \begin{equation*}
        c_i = \frac{i!}{d!} \sum_{j=0}^i \frac{(d - j)!}{(i - j)!} a_j.
    \end{equation*}
\end{exercise}

\begin{solution}
    Lemma~1.2 states that for $j = 0, 1, \ldots, d$,
    \begin{equation*}
        x^j = \frac{(d - j)!}{d!} \sum_{i = j}^d \frac{i!}{(i - j)!} B_i^d(x).
    \end{equation*}

    We have that
    \begin{gather*}
        \sum_{j = 0}^d a_j x^j = \sum_{i = 0}^d c_i B_i^d(x) \\
        \sum_{j = 0}^d a_j \left[
            \frac{(d - j)!}{d!} \sum_{i = j}^d \frac{i!}{(i - j)!} B_i^d(x)
        \right]
        = \sum_{i = 0}^d c_i B_i^d(x) \\
    \end{gather*}
    As we have $i \geq j$, we can reorder the summation to the form $j \leq i$, by using
    \begin{equation*}
        \sum_{j = 0}^d \sum_{i = j}^d (\ldots) = \sum_{i = 0}^d \sum_{j = 0}^i (\ldots).
    \end{equation*}
    This gives us
    \begin{equation*}
        \sum_{i = 0}^d \left[
            \sum_{j = 0}^i a_j \frac{(d - j)!}{d!} \frac{i!}{(i - j)!}
        \right] B_i^d(x) = \sum_{i = 0}^d c_i B_i^d(x).
    \end{equation*}
    Which by isolating the coefficients, gives us
    \begin{equation*}
        c_i = \frac{i!}{d!} \sum_{j = 0}^i \frac{(d - j)!}{(i - j)!} a_j,
    \end{equation*}
    as we wanted to show.
\end{solution}

\end{document}