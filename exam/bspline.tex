\section{B-Splines}
A B-spline is defined from a set of knots
\begin{equation}
    \boldsymbol{t} = (t_1, t_2, \dots, t_{n + d + 1}),
\end{equation}
where $n$ and $d$ are respectively the number and degree of the B-splines.
We assume that the knotvector satisfies
\begin{equation}
    t_{i} \leq t_{i + 1}
    \quad\text{and}\quad
    t_{i} < t_{i + d + 1}.
\end{equation}
The $i$-th B-spline is then defined by
\begin{equation}
    B_{i, d, \boldsymbol{t}}(x)
    = B[t_i, t_{i + 1}, \dots, t_{i + d + 1}](x)
    = [t_i, t_{i + 1}, \dots, t_{i + d + 1}](\cdot - x)_+^d,
\end{equation}
where we here use the divided differences of the truncated power.

B-splines satisfy the property that $B_{i, d, \boldsymbol{t}}(x)$ is zero for all $x \notin [t_i, t_{i + d + 1}]$.
We can see this simply for $x > t_{i + d + 1}$, as then
\begin{equation}
    (t_j - x)_+^d = 0,
    \qquad
    j = i, i + 1, \dots, i + d + 1.
\end{equation}
In the other case, where $x < t_i$, we need an important result for divided differences.
One definition for divided differences is that for a function $f$ defined at points $x_0, x_1, \dots, x_k$, the divided difference of $f$ is the leading coefficient in monomial form of the interpolating polynomial.
Importantly in our case, for $x < t_i$, we have that
\begin{equation}
    (t_j - x)_+^d = (t_j - x)^d \in \pi_d
    \qquad
    j = i, i + 1, \dots, i + d + 1.
\end{equation}
Of important note is that we have $d + 2$ points, such that the interpolating polynomial has a zero coefficient in the leading term.

We also have translation invariance of the B-splines, i.e.\ that
\begin{equation}
    B[t_i + y, t_{i + 1} + y, \dots, t_{i + d + 1} + y](x)
    = B[t_i, t_{i + 1}, \dots, t_{i + d + 1}](x - y).
\end{equation}
We can see this as
\begin{equation}
    ((t + y) - x)_+^d = (t - (x - y))_+^d.
\end{equation}

In the simples case, where $d = 0$, we have
\begin{equation}
    B_i,0(x) = -(t_i - x)_+^d + (t_{i + 1} - x)_+^d,
\end{equation}
showing clearly that the $B_{i, 0}$ is only non-zero on the interval $[t_i, t_{i + 1}]$, where it is equal to one.

In the case of an arbitrary function $f$, we that
\begin{equation}
    [x_0, x_1, \dots, x_{k}]f
    = \sum_{i = 0}^{k} \frac{
        f(x_i)
    }{
        \prod_{\substack{j = 0 \\ j \neq i}}^{k} (x_i - x_j)
    }.
\end{equation}
This allows us to write explicitly
\begin{equation}
    B_{i, d}(x) = \sum_{j = i}^{i + d + 1} a_j (t_j - x)_+^d,
\end{equation}
with
\begin{equation}
    a_j = \frac{t_{i + d + 1} - t_i}{\prod_{\substack{k = i \\ j \neq i}}^{i + d + 1} (t_j - t_k)}.
\end{equation}