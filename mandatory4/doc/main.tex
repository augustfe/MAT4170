\documentclass[a4paper,12pt]{article}

\usepackage[utf8]{inputenc}
\usepackage[T1]{fontenc}
\usepackage{microtype}
\usepackage{amsmath, amssymb}
\usepackage{graphicx}
\usepackage{geometry}

\usepackage{hyperref}
\usepackage{xcolor}
\hypersetup{ % this is just my personal choice, feel free to change things
    colorlinks,
    linkcolor={red!50!black},
    citecolor={blue!50!black},
    urlcolor={blue!80!black},
}
\usepackage[capitalise]{cleveref}

\geometry{margin=1in}

\title{Mandatory Assignment 4}
\author{August Femtehjell}
\date{\today}

\begin{document}

\maketitle

\section{Problem 1: Modelling cross sections of a heart}
We are given nine \verb|.dat| files, containing the contours of a heart at different levels.
\verb|hj1.dat| is at the bottom, while \verb|hj9.dat| is at the top.
Each file contains values of the form $(x_i, y_i, z_i)_{i = 1}^n$, where $z_i$ is constant for each file.

We firstly need to order the data, such that we have $(u_i, x_i, y_i, z_i)_{i = 1}^n$, where $(u_i)_{i = 1}^n$ is an increasing sequence with $u_1 = 0$.
We do this through chord length parametrization, setting
\begin{equation}
    u_i = u_{i - 1} + \sqrt{
        (x_i - x_{i - 1})^2 + (y_i - y_{i - 1})^2 + (z_i - z_{i - 1})^2
    }
\end{equation}
for $i = 2, \ldots, n$.
We then approximate the data by a cubic splines, using a least squares approach.

We choose the knots to be uniformly distributed, with 5\%, 10\% and 20\% of the number of data points.
As we are seeking cubic splines, we have $d = 3$, and seek the splines
\begin{equation}
    g(x) = \sum_{i = 1}^{n} c_j B_{j, d, \mathbf{t}}(x)
\end{equation}
such that
\begin{equation}
    \sum_{i = 1}^{m} \left( g(x_i) - y_i \right)^2
\end{equation}
is minimized.
With
\begin{equation}
    A
    = \left[ B_{j, d, \mathbf{t}}(x_i) \right]_{i,j = 1}^{m,n}
    =
    \begin{bmatrix}
        B_1(x_1) & \cdots & B_n(x_1) \\
        \vdots & \ddots & \vdots \\
        B_1(x_m) & \cdots & B_n(x_m)
    \end{bmatrix}
\end{equation}
and letting $\mathbf{c} = [c_1, \ldots, c_n]^T \in \mathbb{R}^n$ and $\mathbf{y} = [y_1, \ldots, y_m]^T \in \mathbb{R}^m$, the system we need to solve becomes
\begin{equation}
    (A^T A) \mathbf{c} = A^T \mathbf{y},
\end{equation}
which we solve by inverting $A^T A$.

Does it then make sense to space the knots with $t_1 = u_1 = 0$ and $t_{n+d+1} = u_n$?

\end{document}